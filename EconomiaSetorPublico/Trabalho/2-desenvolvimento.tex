\section{Desenvolvimento}


\subsection{Finanças Públicas}
De acordo com Gruber \cite{pfpp}, finanças públicas é o estudo do papel do governo na economia. Desta forma, busca-se responder aos seguintes questionamentos, quando o governo deve intervir ? Como ele pode intervir ? Qual o efeito dessa intervenção no resultado da economia ? Por quê os governos escolhem intervir da maneira como fazem ?

Um conceito importante que também deve ser apresentado é o de externalidades, isto porque, elas representam um falha no mercado que justifica a intervenção dos governos na economia.

Ainda segundo Gruber, externalidades surgem quando as ações de um grupo fazem as condições de outro grupo melhorarem ou piorarem, sem que os primeiros recebem os custos ou benefícios de suas ações. 


\subsection{Quando intervir}
Desde os primeiros momentos da pandemia no Brasil, um forte debate sobre o dilema de preservação do serviço público de saúde para o enfretamento à Covid-19 e o colapso da economia devido à política de \emph{lockdown}  determinada por estados e municípios.

Um dos motivos que ensejam a intervenção do estado da economia é quando o mercado apresenta uma falha, nesse caso ocasionada por uma \emph{externalidade negativa}. Segundo dados do ipea \cite{ipea} o setor público consolidado (SPC) registrou um deficit nominal de 13,7\% do PIB em 2020, todo ele atribuível ao resultado do governo federal, já que para estados e municípios e empresas estatais o deficit nominal foi zero.

O impacto sobre a atividade econômica e  a queda da arrecadação de impostos e outras receitas ligadas ao ciclo econômico explicam boa parte desse déficit no âmbito federal.
% TODO impacto social no mercado de trabalho

Riani \cite{riani} considera que o envolvimento do governo nessa área faz-se necessário, devido principalmente aos aspectos sociais envolvidos. Dessa forma, o governo poderia atuar no problemas das externalidades de diversas maneiras.


\subsection{Como intervir}
O governo surge como alternativa para criação de mecanismos que visem minimizar o desequílibrio na redistribuição do valor gerado pela atividade produtiva do país. Um desses mecanismos possíveis seriam as transferências de rendas aos cidadãos.

O programa Renda Básica Emergencial (RBE) representa um choque de
política que se adiciona a essa trajetória de referência da economia. Dada a estrutura do modelo e sua base de dados, esse choque reflete um aumento de transferências do governo a grupos específicos de famílias, o que deve ser mapeado de acordo com as regras do programa e a base de dados do modelo. 

O auxílio emergencial incluía, entre os elegíveis ao benefício, trabalhadores informais, autônomos, trabalhadores temporários, microempreendedores individuais, além de  beneficiários dos programas de transferência de renda, como o Bolsa Família (BF). O benefício de R\$ 600 foi definido incialmente por três meses, mas foi prorrogado até dezembro de 2020: até setembro, no valor de R\$ 600; e de outubro a dezembro, no valor de R\$ 300. Eram elegíveis ao recebimento pessoas maiores de 18 anos, com renda de até meio salário mínimo (SM) per capita ou renda familiar de até três SMs, limitado a duas cotas por família, sendo que as mulheres provedoras de famílias monoparentais podiam receber duas cotas do auxílio. À exclusão do BF, o indivíduo não poderia ter benefício previdenciário ou assistencial \cite{auxilio}.


% #TODO citação ao site

\subsection{Efeitos da Intervenção}

\subsubsection{Efeitos Diretos}

Dado a alta elasticidade renda-consumo das classes menos favorecidas, simulações conduzidas pelo IPEA reforçam o pressuposto que o aumento da renda é integralmente convertido em consumo.

O estímulo à atividade econômica causa um crescimento da atividade produtiva,  criando a necessidade de mais insumos e ocasionando um efeito na renda dos fatores, trabalho e capital. Esse desequilíbrio tem impactos no nível de atividade, consumo das famílias, investimento e emprego. Por fim, o crescimento da arrecadação de impostos é observado como consequência dessa medida.

\subsubsection{Efeitos Indiretos}

Embora o programa seja voltado para as famílias mais vulneráveis, é perceptível o impacto nas classes mais altas, mesmo que de forma indireta. A transferência de renda focada nas classes mais pobres acaba por gerar renda nas classes mais ricas, como resultado do efeito direto na atividade econômica. Importante citar que o efeito imediato segue o calendário de pagamento do benefício.


Inegavelmente vale apontar a importância da RBE para o enfrentamento aos efeitos da crise de pandemia de Covid-19 na renda das famílias, principalmente aquelas em que a atividade econômica é marcada pela informalidade.

 \subsection{Motivação}

 Historicamente os programas sociais de transferência de renda no Brasil buscam interromper o ciclo da pobreza com iniciativas que visam retirar os filhos de famílias pobres da rua ou do trabalho precoce para ir para a escola, em troca de compensação financeira.

 Um programa social é a unidade mínima de alocação de recursos que, através de um conjunto integrado de atividades pretende transformar uma parcela da realidade, reduzindo ou eliminando um déficit, ou solucionando um problema. Cada projeto tem uma população-objetivo, espacialmente localizada, que deveria receber seus benefícios (COMISSÃO ECONÔMICA PARA AMÉRICA LATINA E CARIBE, 1997).
 
 Podemos elencar fome e pobreza como fatores relevantes que balizam a implantação de políticas de transferência de renda, medida esta que configura-se condicionante à sobrevivência, objetivando sempre a equalização do nível de desigualdade de miséria.
 
 A tabela 1 evidencia parcela importante ($\simeq 40\%$) da população economicamente ativa se encontra nos setores que extraem seus rendimentos diretamente da venda de produtos a pessoas, prestação de serviços ao público e a empresas. Esses setores são diretamente impactados por uma quarentena prolongada em decorrência da Covid-19.


\begin{table}[ht]
\centering
\begin{tabular}{lll}
\hline
           & FREQ.      & \%      \\
\hline
Vulnerável & 37.047.444 & 40,1\%  \\
Outros     & 55.285.425 & 59,9\%  \\
\hline
Total      & 92.332.869 & 100,0\%
\end{tabular}
\caption[fonte]{Fonte: \cite{kawaoka}}
\end{table}

