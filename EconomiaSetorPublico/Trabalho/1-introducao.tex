\section{Introdução}

Ao final de 2019, em uma cidade chinesa chamada Wuhan, situada na província de Hubei, após vários casos de pneumonia, as autoridades chinesas confirmaram que haviam identificado um novo tipo de coronavírus até então não identificado em humanos. Essa nova cepa recebeu o nome de SARS-CoV-2 e é responsável por causar a doença Covid-19.
Em janeiro de 2020, a OMS (sigla) declarou que o surto do novo coronavírus constitui uma Emergência de Saúde Pública de Importância Internacional (ESPII) \cite{paho}. 

No Brasil, os primeiros casos foram confirmados  no mês de fevereiro. Desde então, diversas ações governamentais foram tomadas no intuito de conter a propagação do vírus, muitas delas com impactos ecômicos e financeiros que afetam diretamente a maioria dos cidadãos brasileiros.

Os governos nacionais estabeleceram diversas medidas para tentar conter a escalada de novos casos e uma das principais delas e adotada na maioria dos países com casos confirmados é a medida não farmacológica do isolamento social, implementado com graus diversos de rigidez \cite{kawaoka}.

Este trabalho tem como objetivo apresentar uma visão geral, à luz da Economia do Setor Público, a política de transfêrencia de renda denominada Renda Básica Emergencial (RBE). Esta política representará uma geração de custos razoáveis para governos, empresas e famílias. Essa medida auxiliará as famílias evitando que deixem de realizar as atividades que lhes geram renda e meios de sustento.

