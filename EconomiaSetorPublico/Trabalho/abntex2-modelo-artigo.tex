%% abtex2-modelo-artigo.tex, v-1.9.7 laurocesar
%% Copyright 2012-2018 by abnTeX2 group at http://www.abntex.net.br/ 
%%
%% This work may be distributed and/or modified under the
%% conditions of the LaTeX Project Public License, either version 1.3
%% of this license or (at your option) any later version.
%% The latest version of this license is in
%%   http://www.latex-project.org/lppl.txt
%% and version 1.3 or later is part of all distributions of LaTeX
%% version 2005/12/01 or later.
%%
%% This work has the LPPL maintenance status `maintained'.
%% 
%% The Current Maintainer of this work is the abnTeX2 team, led
%% by Lauro César Araujo. Further information are available on 
%% http://www.abntex.net.br/
%%
%% This work consists of the files abntex2-modelo-artigo.tex and
%% abntex2-modelo-references.bib
%%

% ------------------------------------------------------------------------
% ------------------------------------------------------------------------
% abnTeX2: Modelo de Artigo Acadêmico em conformidade com
% ABNT NBR 6022:2018: Informação e documentação - Artigo em publicação 
% periódica científica - Apresentação
% ------------------------------------------------------------------------
% ------------------------------------------------------------------------

\documentclass[
	% -- opções da classe memoir --
	article,			% indica que é um artigo acadêmico
	11pt,				% tamanho da fonte
	oneside,			% para impressão apenas no recto. Oposto a twoside
	a4paper,			% tamanho do papel. 
	% -- opções da classe abntex2 --
	%chapter=TITLE,		% títulos de capítulos convertidos em letras maiúsculas
	%section=TITLE,		% títulos de seções convertidos em letras maiúsculas
	%subsection=TITLE,	% títulos de subseções convertidos em letras maiúsculas
	%subsubsection=TITLE % títulos de subsubseções convertidos em letras maiúsculas
	% -- opções do pacote babel --
	english,			% idioma adicional para hifenização
	brazil,				% o último idioma é o principal do documento
	sumario=tradicional
	]{abntex2}


% ---
% PACOTES
% ---

% ---
% Pacotes fundamentais 
% ---
\usepackage{lmodern}			% Usa a fonte Latin Modern
\usepackage[T1]{fontenc}		% Selecao de codigos de fonte.
\usepackage[utf8]{inputenc}		% Codificacao do documento (conversão automática dos acentos)
\usepackage{indentfirst}		% Indenta o primeiro parágrafo de cada seção.
\usepackage{nomencl} 			% Lista de simbolos
\usepackage{color}				% Controle das cores
\usepackage{graphicx}			% Inclusão de gráficos
\usepackage{microtype} 			% para melhorias de justificação

% ---
		
% ---
% Pacotes adicionais, usados apenas no âmbito do Modelo Canônico do abnteX2
% ---
\usepackage{lipsum}				% para geração de dummy text
% ---
		
% ---
% Pacotes de citações
% ---
\usepackage[brazilian,hyperpageref]{backref}	 % Paginas com as citações na bibl
\usepackage[alf]{abntex2cite}	% Citações padrão ABNT
% ---

% ---
% Configurações do pacote backref
% Usado sem a opção hyperpageref de backref
\renewcommand{\backrefpagesname}{Citado na(s) página(s):~}
% Texto padrão antes do número das páginas
\renewcommand{\backref}{}
% Define os textos da citação
\renewcommand*{\backrefalt}[4]{
	\ifcase #1 %
		Nenhuma citação no texto.%
	\or
		Citado na página #2.%
	\else
		Citado #1 vezes nas páginas #2.%
	\fi}%
% ---

% --- Informações de dados para CAPA e FOLHA DE ROSTO ---
\titulo{Estudo de Caso do Auxílio Emergencial no Contexto da Pandemia de Covid-19}
\tituloestrangeiro{}

\autor{
Aguimar Neto\thanks{``Auditor Fiscal de TI na Sefaz/CE, aguimar@gmail.com''} 
\\[0.5cm]
Bruno Machado\thanks{``Auditor Fiscal de TI na Sefaz/CE, brunofmac@gmail.com''} }
\local{Brasil}
\data{\today}
% ---

% ---
% Configurações de aparência do PDF final

% alterando o aspecto da cor azul
\definecolor{blue}{RGB}{41,5,195}

% informações do PDF
\makeatletter
\hypersetup{
     	%pagebackref=true,
		pdftitle={\@title}, 
		pdfauthor={\@author},
    	pdfsubject={Modelo de artigo científico com abnTeX2},
	    pdfcreator={LaTeX with abnTeX2},
		pdfkeywords={abnt}{latex}{abntex}{abntex2}{atigo científico}, 
		colorlinks=true,       		% false: boxed links; true: colored links
    	linkcolor=blue,          	% color of internal links
    	citecolor=blue,        		% color of links to bibliography
    	filecolor=magenta,      		% color of file links
		urlcolor=blue,
		bookmarksdepth=4
}
\makeatother
% --- 

% ---
% compila o indice
% ---
\makeindex
% ---

% ---
% Altera as margens padrões
% ---
\setlrmarginsandblock{3cm}{3cm}{*}
\setulmarginsandblock{3cm}{3cm}{*}
\checkandfixthelayout
% ---

% --- 
% Espaçamentos entre linhas e parágrafos 
% --- 

% O tamanho do parágrafo é dado por:
\setlength{\parindent}{1.3cm}

% Controle do espaçamento entre um parágrafo e outro:
\setlength{\parskip}{0.2cm}  % tente também \onelineskip

% Espaçamento simples
\SingleSpacing


% ----
% Início do documento
% ----
\begin{document}

% Seleciona o idioma do documento (conforme pacotes do babel)
%\selectlanguage{english}
\selectlanguage{brazil}

% Retira espaço extra obsoleto entre as frases.
\frenchspacing 

% ----------------------------------------------------------
% ELEMENTOS PRÉ-TEXTUAIS
% ----------------------------------------------------------

%---
%
% Se desejar escrever o artigo em duas colunas, descomente a linha abaixo
% e a linha com o texto ``FIM DE ARTIGO EM DUAS COLUNAS''.
% \twocolumn[    		% INICIO DE ARTIGO EM DUAS COLUNAS
%
%---

% página de titulo principal (obrigatório)
\maketitle

% titulo em outro idioma (opcional)



% resumo em português
\begin{resumoumacoluna}
  Neste trabalho, nós temos o objetivo de mostrar como os governos podem e devem interferir na economia. Pegamos como exemplo o programa de transfêrencia de renda denominado Renda Básica Emergencial e percorremos alguns conceitos utilizados em Economia do Setor Público. Definimos alguns conceitos e analisamos os efeitos e como foi feita a intervenção. Por fim esclarecemos as conclusões que podem ser extraídas de tal medida economica.
 \vspace{\onelineskip}
 
 \noindent
 \textbf{Palavras-chave}: intervenção. economia. governo. efeitos.
\end{resumoumacoluna}


% resumo em inglês
% \renewcommand{\resumoname}{Abstract}
% \begin{resumoumacoluna}
%  \begin{otherlanguage*}{english}
%    According to ABNT NBR 6022:2018, an abstract in foreign language is optional.

%    \vspace{\onelineskip}
 
%    \noindent
%    \textbf{Keywords}: latex. abntex.
%  \end{otherlanguage*}  
% \end{resumoumacoluna}

% ]  				% FIM DE ARTIGO EM DUAS COLUNAS
% ---

% \begin{center}\smaller
% \textbf{Data de submissão e aprovação}: elemento obrigatório. Indicar dia, mês e ano


% \textbf{Identificação e disponibilidade}: Mestrando no curso Economia do Setor Público - UFC, aguimar@gmail.com
% \end{center}

\newpage

% ----------------------------------------------------------
% ELEMENTOS TEXTUAIS
% ----------------------------------------------------------
\textual

% ----------------------------------------------------------
% Introdução
% ----------------------------------------------------------
\section{Introdução}

Ao final de 2019, em uma cidade chinesa chamada Wuhan, situada na província de Hubei, após vários casos de pneumonia, as autoridades chinesas confirmaram que haviam identificado um novo tipo de coronavírus até então não identificado em humanos. Essa nova cepa recebeu o nome de SARS-CoV-2 e é responsável por causar a doença Covid-19.
Em janeiro de 2020, a OMS (sigla) declarou que o surto do novo coronavírus constitui uma Emergência de Saúde Pública de Importância Internacional (ESPII) \cite{paho}. 

No Brasil, os primeiros casos foram confirmados  no mês de fevereiro. Desde então, diversas ações governamentais foram tomadas no intuito de conter a propagação do vírus, muitas delas com impactos ecômicos e financeiros que afetam diretamente a maioria dos cidadãos brasileiros.

Os governos nacionais estabeleceram diversas medidas para tentar conter a escalada de novos casos e uma das principais delas e adotada na maioria dos países com casos confirmados é a medida não farmacológica do isolamento social, implementado com graus diversos de rigidez \cite{kawaoka}.

Este trabalho tem como objetivo apresentar uma visão geral, à luz da Economia do Setor Público, a política de transfêrencia de renda denominada Renda Básica Emergencial (RBE). Esta política representará uma geração de custos razoáveis para governos, empresas e famílias. Essa medida auxiliará as famílias evitando que deixem de realizar as atividades que lhes geram renda e meios de sustento.




% ----------------------------------------------------------
% Desenvolvimento
% ----------------------------------------------------------
\section{Desenvolvimento}


\subsection{Finanças Públicas}
De acordo com Gruber \cite{pfpp}, finanças públicas é o estudo do papel do governo na economia. Desta forma, busca-se responder aos seguintes questionamentos, quando o governo deve intervir ? Como ele pode intervir ? Qual o efeito dessa intervenção no resultado da economia ? Por quê os governos escolhem intervir da maneira como fazem ?

Um conceito importante que também deve ser apresentado é o de externalidades, isto porque, elas representam um falha no mercado que justifica a intervenção dos governos na economia.

Ainda segundo Gruber, externalidades surgem quando as ações de um grupo fazem as condições de outro grupo melhorarem ou piorarem, sem que os primeiros recebem os custos ou benefícios de suas ações. 


\subsection{Quando intervir}
Desde os primeiros momentos da pandemia no Brasil, um forte debate sobre o dilema de preservação do serviço público de saúde para o enfretamento à Covid-19 e o colapso da economia devido à política de \emph{lockdown}  determinada por estados e municípios.

Um dos motivos que ensejam a intervenção do estado da economia é quando o mercado apresenta uma falha, nesse caso ocasionada por uma \emph{externalidade negativa}. Segundo dados do ipea \cite{ipea} o setor público consolidado (SPC) registrou um deficit nominal de 13,7\% do PIB em 2020, todo ele atribuível ao resultado do governo federal, já que para estados e municípios e empresas estatais o deficit nominal foi zero.

O impacto sobre a atividade econômica e  a queda da arrecadação de impostos e outras receitas ligadas ao ciclo econômico explicam boa parte desse déficit no âmbito federal.
% TODO impacto social no mercado de trabalho

Riani \cite{riani} considera que o envolvimento do governo nessa área faz-se necessário, devido principalmente aos aspectos sociais envolvidos. Dessa forma, o governo poderia atuar no problemas das externalidades de diversas maneiras.


\subsection{Como intervir}
O governo surge como alternativa para criação de mecanismos que visem minimizar o desequílibrio na redistribuição do valor gerado pela atividade produtiva do país. Um desses mecanismos possíveis seriam as transferências de rendas aos cidadãos.

O programa Renda Básica Emergencial (RBE) representa um choque de
política que se adiciona a essa trajetória de referência da economia. Dada a estrutura do modelo e sua base de dados, esse choque reflete um aumento de transferências do governo a grupos específicos de famílias, o que deve ser mapeado de acordo com as regras do programa e a base de dados do modelo. 

O auxílio emergencial incluía, entre os elegíveis ao benefício, trabalhadores informais, autônomos, trabalhadores temporários, microempreendedores individuais, além de  beneficiários dos programas de transferência de renda, como o Bolsa Família (BF). O benefício de R\$ 600 foi definido incialmente por três meses, mas foi prorrogado até dezembro de 2020: até setembro, no valor de R\$ 600; e de outubro a dezembro, no valor de R\$ 300. Eram elegíveis ao recebimento pessoas maiores de 18 anos, com renda de até meio salário mínimo (SM) per capita ou renda familiar de até três SMs, limitado a duas cotas por família, sendo que as mulheres provedoras de famílias monoparentais podiam receber duas cotas do auxílio. À exclusão do BF, o indivíduo não poderia ter benefício previdenciário ou assistencial \cite{auxilio}.


% #TODO citação ao site

\subsection{Efeitos da Intervenção}

\subsubsection{Efeitos Diretos}

Dado a alta elasticidade renda-consumo das classes menos favorecidas, simulações conduzidas pelo IPEA reforçam o pressuposto que o aumento da renda é integralmente convertido em consumo.

O estímulo à atividade econômica causa um crescimento da atividade produtiva,  criando a necessidade de mais insumos e ocasionando um efeito na renda dos fatores, trabalho e capital. Esse desequilíbrio tem impactos no nível de atividade, consumo das famílias, investimento e emprego. Por fim, o crescimento da arrecadação de impostos é observado como consequência dessa medida.

\subsubsection{Efeitos Indiretos}

Embora o programa seja voltado para as famílias mais vulneráveis, é perceptível o impacto nas classes mais altas, mesmo que de forma indireta. A transferência de renda focada nas classes mais pobres acaba por gerar renda nas classes mais ricas, como resultado do efeito direto na atividade econômica. Importante citar que o efeito imediato segue o calendário de pagamento do benefício.


Inegavelmente vale apontar a importância da RBE para o enfrentamento aos efeitos da crise de pandemia de Covid-19 na renda das famílias, principalmente aquelas em que a atividade econômica é marcada pela informalidade.

 \subsection{Motivação}

 Historicamente os programas sociais de transferência de renda no Brasil buscam interromper o ciclo da pobreza com iniciativas que visam retirar os filhos de famílias pobres da rua ou do trabalho precoce para ir para a escola, em troca de compensação financeira.

 Um programa social é a unidade mínima de alocação de recursos que, através de um conjunto integrado de atividades pretende transformar uma parcela da realidade, reduzindo ou eliminando um déficit, ou solucionando um problema. Cada projeto tem uma população-objetivo, espacialmente localizada, que deveria receber seus benefícios (COMISSÃO ECONÔMICA PARA AMÉRICA LATINA E CARIBE, 1997).
 
 Podemos elencar fome e pobreza como fatores relevantes que balizam a implantação de políticas de transferência de renda, medida esta que configura-se condicionante à sobrevivência, objetivando sempre a equalização do nível de desigualdade de miséria.
 
 A tabela 1 evidencia parcela importante ($\simeq 40\%$) da população economicamente ativa se encontra nos setores que extraem seus rendimentos diretamente da venda de produtos a pessoas, prestação de serviços ao público e a empresas. Esses setores são diretamente impactados por uma quarentena prolongada em decorrência da Covid-19.


\begin{table}[ht]
\centering
\begin{tabular}{lll}
\hline
           & FREQ.      & \%      \\
\hline
Vulnerável & 37.047.444 & 40,1\%  \\
Outros     & 55.285.425 & 59,9\%  \\
\hline
Total      & 92.332.869 & 100,0\%
\end{tabular}
\caption[fonte]{Fonte: \cite{kawaoka}}
\end{table}



% ----------------------------------------------------------
% Seção de explicações
% ----------------------------------------------------------
% \section{Exemplos de comandos}

% \subsection{Margens}

% A norma ABNT NBR 6022:2018 não estabelece uma margem específica a ser utilizada
% no artigo científico. Dessa maneira, caso deseje alterar as margens, utilize os
% comandos abaixo:

% \begin{verbatim}
%    \setlrmarginsandblock{3cm}{3cm}{*}
%    \setulmarginsandblock{3cm}{3cm}{*}
%    \checkandfixthelayout
% \end{verbatim}

% \subsection{Duas colunas}

% É comum que artigos científicos sejam escritos em duas colunas. Para isso,
% adicione a opção \texttt{twocolumn} à classe do documento, como no exemplo:

% \begin{verbatim}
%    \documentclass[article,11pt,oneside,a4paper,twocolumn]{abntex2}
% \end{verbatim}

% É possível indicar pontos do texto que se deseja manter em apenas uma coluna,
% geralmente o título e os resumos. Os resumos em única coluna em documentos com
% a opção \texttt{twocolumn} devem ser escritos no ambiente
% \texttt{resumoumacoluna}:

% \begin{verbatim}
%    \twocolumn[              % INICIO DE ARTIGO EM DUAS COLUNAS

%      \maketitle             % pagina de titulo

%      \renewcommand{\resumoname}{Nome do resumo}
%      \begin{resumoumacoluna}
%         Texto do resumo.
      
%         \vspace{\onelineskip}
 
%         \noindent
%         \textbf{Palavras-chave}: latex. abntex. editoração de texto.
%      \end{resumoumacoluna}
   
%    ]                        % FIM DE ARTIGO EM DUAS COLUNAS
% \end{verbatim}

% \subsection{Recuo do ambiente \texttt{citacao}}

% Na produção de artigos (opção \texttt{article}), pode ser útil alterar o recuo
% do ambiente \texttt{citacao}. Nesse caso, utilize o comando:

% \begin{verbatim}
%    \setlength{\ABNTEXcitacaorecuo}{1.8cm}
% \end{verbatim}

% Quando um documento é produzido com a opção \texttt{twocolumn}, a classe
% \textsf{abntex2} automaticamente altera o recuo padrão de 4 cm, definido pela
% ABNT NBR 10520:2002 seção 5.3, para 1.8 cm.

% \section{Cabeçalhos e rodapés customizados}

% Diferentes estilos de cabeçalhos e rodapés podem ser criados usando os
% recursos padrões do \textsf{memoir}.

% Um estilo próprio de cabeçalhos e rodapés pode ser diferente para páginas pares
% e ímpares. Observe que a diferenciação entre páginas pares e ímpares só é
% utilizada se a opção \texttt{twoside} da classe \textsf{abntex2} for utilizado.
% Caso contrário, apenas o cabeçalho padrão da página par (\emph{even}) é usado.

% Veja o exemplo abaixo cria um estilo chamado \texttt{meuestilo}. O código deve
% ser inserido no preâmbulo do documento.

% \begin{verbatim}
% %%criar um novo estilo de cabeçalhos e rodapés
% \makepagestyle{meuestilo}
%   %%cabeçalhos
%   \makeevenhead{meuestilo} %%pagina par
%      {topo par à esquerda}
%      {centro \thepage}
%      {direita}
%   \makeoddhead{meuestilo} %%pagina ímpar ou com oneside
%      {topo ímpar/oneside à esquerda}
%      {centro\thepage}
%      {direita}
%   \makeheadrule{meuestilo}{\textwidth}{\normalrulethickness} %linha
%   %% rodapé
%   \makeevenfoot{meuestilo}
%      {rodapé par à esquerda} %%pagina par
%      {centro \thepage}
%      {direita} 
%   \makeoddfoot{meuestilo} %%pagina ímpar ou com oneside
%      {rodapé ímpar/onside à esquerda}
%      {centro \thepage}
%      {direita}
% \end{verbatim}

% Para usar o estilo criado, use o comando abaixo imediatamente após um dos
% comandos de divisão do documento. Por exemplo:

% \begin{verbatim}
%    \begin{document}
%      %%usar o estilo criado na primeira página do artigo:
%      \pretextual
%      \pagestyle{meuestilo}
     
%      \maketitle
%      ...
     
%      %%usar o estilo criado nas páginas textuais
%      \textual
%      \pagestyle{meuestilo}
     
%      \chapter{Novo capítulo}
%      ...
%    \end{document}  
% \end{verbatim}
   
% Outras informações sobre cabeçalhos e rodapés estão disponíveis na seção 7.3 do
% manual do \textsf{memoir} \cite{memoir}.

% \section{Mais exemplos no Modelo Canônico de Trabalhos Acadêmicos}

% Este modelo de artigo é limitado em número de exemplos de comandos, pois são
% apresentados exclusivamente comandos diretamente relacionados com a produção de
% artigos.

% Para exemplos adicionais de \abnTeX\ e \LaTeX, como inclusão de figuras,
% fórmulas matemáticas, citações, e outros, consulte o documento
% \citeonline{abntex2modelo}.

% \section{Consulte o manual da classe \textsf{abntex2}}

% Consulte o manual da classe \textsf{abntex2} \cite{abntex2classe} para uma
% referência completa das macros e ambientes disponíveis.

% ---
% Finaliza a parte no bookmark do PDF, para que se inicie o bookmark na raiz
% ---
\bookmarksetup{startatroot}% 
% ---

% ---
% Conclusão
% ---

\section{Considerações Finais}

O estudo superficial da política de transferência de renda mostra a necessidade de se pensar em ações de enfretamento focalizadas nos mais pobres, que absorvem o efeito mais pronunciado de uma redução da atividade econômica e, consequentemente, do emprego. Outras medidas são necessárias quando levamos em conta as vulnerabilidades das classes mais baixas de renda, que dependem fundamentalmente de transporte público, vivem em condições precárias e têm menor acesso a serviços de saúde.

Segundo Débora Freire Cardoso \cite{debora}, mudanças na estrutura distributiva trazem consigo o potencial de promover alterações importantes no consumo das famílias , tendo em vista a incorporação de famílias de menor renda ao mercado consumidor e a ascenção de segmentos de baixa renda à classe média. Alterações importantes na composição do consumo, geram, por sua vez, impactos sobre a estrutura produtiva, modificando a distribuição de fatores produtivos e seus preços relativos.

A pesquisadora demonstra que o impacto assimétrico do programa na estrutura produtiva estimula setores com produção voltada para o mercado interno em detrimento de \textit{commodities} exportadas, acarretando um aumento um aumento na massa salarial apropriada pelas classes mais baixas em proporção maior do que as das classes mais altas de renda, exercendo uma leve tendência de desconcentração das remunerações. 


% \section{Considerações finais}

% \lipsum[1]

% \begin{citacao}
% \lipsum[2]
% \end{citacao}

% \lipsum[3]

% ----------------------------------------------------------
% ELEMENTOS PÓS-TEXTUAIS
% ----------------------------------------------------------
\postextual

% ----------------------------------------------------------
% Referências bibliográficas
% ----------------------------------------------------------
\newpage

\bibliography{abntex2-modelo-references}

% ----------------------------------------------------------
% Glossário
% ----------------------------------------------------------
%
% Há diversas soluções prontas para glossário em LaTeX. 
% Consulte o manual do abnTeX2 para obter sugestões.
%
%\glossary

% ----------------------------------------------------------
% Apêndices
% ----------------------------------------------------------

% ---
% Inicia os apêndices
% ---
% \begin{apendicesenv}

% % ----------------------------------------------------------
% \chapter{Nullam elementum urna vel imperdiet sodales elit ipsum pharetra ligula
% ac pretium ante justo a nulla curabitur tristique arcu eu metus}
% % ----------------------------------------------------------
% \lipsum[55-56]

% \end{apendicesenv}
% ---

% ----------------------------------------------------------
% Anexos
% ----------------------------------------------------------
\cftinserthook{toc}{AAA}
% ---
% Inicia os anexos
% ---
%\anexos
% \begin{anexosenv}

% % ---
% \chapter{Cras non urna sed feugiat cum sociis natoque penatibus et magnis dis
% parturient montes nascetur ridiculus mus}
% % ---

% \lipsum[31]

% \end{anexosenv}

% ----------------------------------------------------------
% Agradecimentos
% ----------------------------------------------------------

% \section*{Agradecimentos}
% Texto sucinto aprovado pelo periódico em que será publicado. Último 
% elemento pós-textual.

\end{document}
