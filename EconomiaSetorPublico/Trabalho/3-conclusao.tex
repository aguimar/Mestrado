\section{Considerações Finais}

O estudo superficial da política de transferência de renda mostra a necessidade de se pensar em ações de enfretamento focalizadas nos mais pobres, que absorvem o efeito mais pronunciado de uma redução da atividade econômica e, consequentemente, do emprego. Outras medidas são necessárias quando levamos em conta as vulnerabilidades das classes mais baixas de renda, que dependem fundamentalmente de transporte público, vivem em condições precárias e têm menor acesso a serviços de saúde.

Segundo Débora Freire Cardoso \cite{debora}, mudanças na estrutura distributiva trazem consigo o potencial de promover alterações importantes no consumo das famílias , tendo em vista a incorporação de famílias de menor renda ao mercado consumidor e a ascenção de segmentos de baixa renda à classe média. Alterações importantes na composição do consumo, geram, por sua vez, impactos sobre a estrutura produtiva, modificando a distribuição de fatores produtivos e seus preços relativos.

A pesquisadora demonstra que o impacto assimétrico do programa na estrutura produtiva estimula setores com produção voltada para o mercado interno em detrimento de \textit{commodities} exportadas, acarretando um aumento um aumento na massa salarial apropriada pelas classes mais baixas em proporção maior do que as das classes mais altas de renda, exercendo uma leve tendência de desconcentração das remunerações. 