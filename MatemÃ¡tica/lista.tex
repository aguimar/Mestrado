Aluno: Aguimar Mendonça Neto

\hypertarget{obtenha-as-derivadas-das-expressuxf5es-abaixo-usando-as-regras-de-derivauxe7uxe3o}{%
\section{1. Obtenha as derivadas das expressões abaixo, usando as regras
de
derivação:}\label{obtenha-as-derivadas-das-expressuxf5es-abaixo-usando-as-regras-de-derivauxe7uxe3o}}

\hypertarget{a-y-x34x2-2sqrtx-x---4}{%
\subsection{\texorpdfstring{(a)
\(y = x^{3}+4x^2-2\sqrt{x} + x - 4\)}{(a) y = x\^{}\{3\}+4x\^{}2-2\textbackslash{}sqrt\{x\} + x - 4}}\label{a-y-x34x2-2sqrtx-x---4}}

\(y'=3x^{3-1} - 2.4x^{2-1}-\frac{1}{2}.2x^{\frac{1}{2}-1}+x^{1-1}-0\)

\(y'=3x^2-8x-\frac{1}{\sqrt{x}}\)+1

\hypertarget{b-fxfrac2x2}{%
\subsection{\texorpdfstring{(b)
\(f(x)=\frac{2}{x^2}\)}{(b) f(x)=\textbackslash{}frac\{2\}\{x\^{}2\}}}\label{b-fxfrac2x2}}

\(f'(x) = -2.2x^{-2-1}\)

\(f'(x)=-\frac{4}{x^3}\)

\hypertarget{c-y-fracx33frac3x2}{%
\subsection{\texorpdfstring{(c)
\(y= \frac{x^3}{3}+\frac{3x}{2}\)}{(c) y= \textbackslash{}frac\{x\^{}3\}\{3\}+\textbackslash{}frac\{3x\}\{2\}}}\label{c-y-fracx33frac3x2}}

\(y' = 3.\frac{x^{3-1}}{3} + \frac{3}{2}x^{1-1}\)

\(y' = x^2+\frac{3}{2}\)

\hypertarget{d-fxx2-3x2-no-ponto-x_02}{%
\subsection{\texorpdfstring{(d) \(f(x)=x^2-3x+2\), no ponto
\(x_0=2\)}{(d) f(x)=x\^{}2-3x+2, no ponto x\_0=2}}\label{d-fxx2-3x2-no-ponto-x_02}}

\(f'(x)=2x^{2-1}+3x^{1-1}+0\)

\(f'(x)=2x+3\)

\(f'(2)=2.2+3=7\)

\hypertarget{efxfracx3102x4}{%
\section{\texorpdfstring{(e)\(f(x)=\frac{x^3+10}{2x^4}\)}{(e)f(x)=\textbackslash{}frac\{x\^{}3+10\}\{2x\^{}4\}}}\label{efxfracx3102x4}}

\hypertarget{regra-do-quociente}{%
\subsubsection{Regra do quociente}\label{regra-do-quociente}}

\([\frac{u(x)}{v(x)}]'=\frac{u'(x).v(x)-u(x).v'(x)}{v(x)^2}\)

\(u(x)=x^3+10\)

\(v(x)=2x^4\)

\(u'(x) = 3x2\)

\(v'(x) = 8x^3\)

\(f'(x)=\frac{3x^2.2x^4-(x^3+10).8x^3}{(2x^4)^2}\)

\(f'(x) = \frac{6x^6-8x^6-80x^3}{4x^8}\)

\(f'(x)=\frac{-2x^3(x^3+40)}{4x^8}\)

\(f'(x)=-\frac{x^3+40}{2x^5}\)

\hypertarget{f-fx3x5-12-x4}{%
\section{\texorpdfstring{(f)
\(f(x)=(3x^5-1)(2-x^4)\)}{(f) f(x)=(3x\^{}5-1)(2-x\^{}4)}}\label{f-fx3x5-12-x4}}

\hypertarget{regra-do-produto}{%
\subsubsection{Regra do produto}\label{regra-do-produto}}

\([u(x).v(x)]'=u'(x).v(x)+u(x).v'(x)\)

\(u(x)=3x^5-1\)

\(v(x)=2-x^4\)

\(u'(x)=15x^4\)

\(v'(x)=-4x^3\)

\(f'(x)=15x^4(2-x^4)+(3x^5-1)(-4x^3)\)

\(f'(x)=30x^4-15x^8-12x^8+4x^3\)

\(f'(x)=-27x^8+30x^4+4x^3\)

\(f'(x)=-x^3(27x^5-30x-4)\)

\hypertarget{g-y6x272}{%
\section{\texorpdfstring{(g)
\(y=(6x^2+7)^2\)}{(g) y=(6x\^{}2+7)\^{}2}}\label{g-y6x272}}

\hypertarget{regra-da-potuxeancia}{%
\subsubsection{Regra da potência}\label{regra-da-potuxeancia}}

\([u(x)^n]'=n.u(x)^{n-1}.u'(x)\)

\(n=2\)

\(u(x)=6x^2+7\)

\(u'(x)=12x\)

\(f'(x)=2(6x^2+7).12x\)

\(f'(x)=144x^3+168x\)

\(f'(x)=24x(6x^2+7)\)

\#(h) \(y=e^x+3x^2e^{2x}\)

\hypertarget{regra-da-funuxe7uxe3o-exponencial}{%
\subsubsection{Regra da função
exponencial}\label{regra-da-funuxe7uxe3o-exponencial}}

\([e^x]'=e^x\)

\([e^{u(x)}]'=e^{u(x)}.u'(x)\)

\hypertarget{regra-do-produto-1}{%
\subsubsection{Regra do produto}\label{regra-do-produto-1}}

\([u(x).v(x)]'=u'(x).v(x)+u(x).v'(x)\)

\(u(x) = 3x^2\)

\(v(x) = e^{2x}\)

\(u'(x) = 6x\)

\(v'(x) = e^{2x}.2\)

\(y'(x)=6x.e^{2x}+3x^2.2e^{2x}\)

\(y'(x)=6x(e^{2x}+xe^{2x})\)

\(y'(x)=6xe^{2x}(1+x)\)

\#(i) \(y=ln(\frac{x^2+2}{e^{-x}})\)

\hypertarget{regra-logaritmo-natual}{%
\subsubsection{Regra logaritmo natual}\label{regra-logaritmo-natual}}

\([ln(x)]'=\frac{1}{x}\)

\hypertarget{regra-da-cadeia}{%
\subsubsection{Regra da cadeia}\label{regra-da-cadeia}}

\([ln(g(x))]'=\frac{1}{g(x)}.g'(x)\)

\(g(x) = (x^2+2)e^{x}\)

\hypertarget{regra-do-produto-2}{%
\paragraph{regra do produto}\label{regra-do-produto-2}}

\(u(x) = x^2+2\)

\(v(x) = e^x\)

\(u'(x) = 2x\)

\(v'(x) = e^x\)

\(g'(x) = 2x.e^x+(x^2+2).e^x\)

\(g'(x)=e^x(x^2+2x+2)\)

\(y'(x)=\frac{1}{(x^2+2)e^{x}}.e^x(x^2+2x+2)\)

\(y'(x) =\frac{x^2+2x+2}{x^2+2}\)

\hypertarget{j-y-lnfracx2x3}{%
\section{\texorpdfstring{(j)
\(y =ln(\frac{x+2}{x^3})\)}{(j) y =ln(\textbackslash{}frac\{x+2\}\{x\^{}3\})}}\label{j-y-lnfracx2x3}}

\(g(x) = (x+2){x^{-3}}\)

\(u(x) = x + 2\)

\(v(x) = x^{-3}\)

\(u'(x) = 1\)

\(v'(x) = -3.x^{-4}\)

\(g'(x) = 1.x^{-3} + (x+2).(-3.x^{-4})\)

\(g'(x) = \frac{1}{x^3} - 3\frac{x+2}{x^4}\)

\(g'(x) = \frac{1}{x^3}(1-(3\frac{x+2}{x}))\)

\(y'(x) = \frac{1}{(x+2){x^{-3}}}.\frac{1}{x^3}(1-(3\frac{x+2}{x}))\)

\(y'(x) = \frac{1}{x+2}-\frac{3}{x}\)

\hypertarget{k-y-sqrt1x2}{%
\section{\texorpdfstring{(k)
\(y = \sqrt{(1+x^2)}\)}{(k) y = \textbackslash{}sqrt\{(1+x\^{}2)\}}}\label{k-y-sqrt1x2}}

\(u(x) = 1 + x^2\)

\(u'(x) = 2x\)

\(y'(x) = \frac{1}{2}.(1+x^2)^{-\frac{1}{2}}.2x\)

\(y'(x) = \frac{x}{\sqrt{1+x^2}}\)

\hypertarget{l-fx-ln2x}{%
\section{\texorpdfstring{(l)
\(f(x) = ln(2x)\)}{(l) f(x) = ln(2x)}}\label{l-fx-ln2x}}

\(g(x)=2x\)

\(g'(x)=2\)

\(f'(x) = \frac{1}{2x}.2\)

\(f'(x) = \frac{1}{x}\)

\hypertarget{encontre-as-derivadas-de-segunda-ordem-das-seguintes-funuxe7uxf5es}{%
\section{2. Encontre as derivadas de segunda ordem das seguintes
funções:}\label{encontre-as-derivadas-de-segunda-ordem-das-seguintes-funuxe7uxf5es}}

\hypertarget{a-y-52---x3---x-10}{%
\subsection{\texorpdfstring{a)
\(y = 5^2 - x^3 - x + 10\)}{a) y = 5\^{}2 - x\^{}3 - x + 10}}\label{a-y-52---x3---x-10}}

\(y'=2.5.x^{2-1} - 3.x^{3-1} - 1.x^{1-1} + 10\)

\(y'=10x-3x^2-1\)

\(y''=1.10.x^{1-1}-2.3.x^{2-1}-0\)

\(y''=10-6x\)

\hypertarget{suponha-que-rx-seja-a-funuxe7uxe3o-de-receita-total-recebida-da-venda-de-x-unidades-de-cadeiras-da-loja-bbc-muxf3veis-e-rx-4x22000x.-calcule-a-receita-marginal-para-x-40.}{%
\section{\texorpdfstring{3. Suponha que \(R(x)\) seja a função de
receita total recebida da venda de x unidades de cadeiras da loja BBC
móveis, e \(R(x)=-4x^2+2000x\). Calcule a receita marginal para x =
40.}{3. Suponha que R(x) seja a função de receita total recebida da venda de x unidades de cadeiras da loja BBC móveis, e R(x)=-4x\^{}2+2000x. Calcule a receita marginal para x = 40.}}\label{suponha-que-rx-seja-a-funuxe7uxe3o-de-receita-total-recebida-da-venda-de-x-unidades-de-cadeiras-da-loja-bbc-muxf3veis-e-rx-4x22000x.-calcule-a-receita-marginal-para-x-40.}}

\hypertarget{receita-marginal-rmg}{%
\subsubsection{Receita Marginal (Rmg)}\label{receita-marginal-rmg}}

A expressão Receita Marginal (Rmg) designa a variação da receita total
(RT) de uma entidade produtiva (uma empresa, por exemplo) provocada pela
variação em uma unidade na produção de determinado bem (Q).

Em termos algébricos: \(Rmg=\frac{ΔRT}{ΔQ}\)

\(R(x)= -4x^2+2000x\)

\(R'(x)=-8x+2000\)

Logo, a Rmg para x = 40 é:

\(R'(40)=-8*40+2000=1680\)

\hypertarget{a-receita-total-de-uma-empresa-uxe9-dada-pela-funuxe7uxe3o-rtq1000q-4q2.-pede-se}{%
\section{\texorpdfstring{5. A receita total de uma empresa é dada pela
função \(RT(Q)=1000Q-4Q^2\).
Pede-se:}{5. A receita total de uma empresa é dada pela função RT(Q)=1000Q-4Q\^{}2. Pede-se:}}\label{a-receita-total-de-uma-empresa-uxe9-dada-pela-funuxe7uxe3o-rtq1000q-4q2.-pede-se}}

\begin{enumerate}
\def\labelenumi{(\alph{enumi})}
\item
  a função receita marginal.
\item
  o valor da receita marginal quando a quantidade é igual a 5.
\end{enumerate}

\(RT(Q)=1000Q-4Q^2\)

\(RT'(Q)=1000-8Q\) (a)

\(RT'(100)=1000-8*100=200\) (b)

\hypertarget{seja-cx-1004x002x2-o-custo-total-de-fabricauxe7uxe3o-de-x-pares-de-caluxe7ados-de-uma-empresa.-determine-o-custo-marginal-quando-x50.}{%
\section{\texorpdfstring{7. Seja \(C(x) = 100+4x+0,02x^2\) o custo total
de fabricação de \(x\) pares de calçados de uma empresa. Determine o
custo marginal quando
x=50.}{7. Seja C(x) = 100+4x+0,02x\^{}2 o custo total de fabricação de x pares de calçados de uma empresa. Determine o custo marginal quando x=50.}}\label{seja-cx-1004x002x2-o-custo-total-de-fabricauxe7uxe3o-de-x-pares-de-caluxe7ados-de-uma-empresa.-determine-o-custo-marginal-quando-x50.}}

\hypertarget{custo-marginal}{%
\subsubsection{Custo Marginal}\label{custo-marginal}}

O custo marginal (Cmg) é um conceito que busca descrever as alterações
causadas no custo total para uma mudança unitária na quantidade
produzida.

Em termos algébricos: \(Cmg=\frac{ΔCT}{ΔQ}\)

\(C(x) = 100+4x+0,02x^2\)

\(C'(x)=4+0,04x\) (\(Cmg\))

\(Cmg(50)=4+0,04*50=6\)

\hypertarget{uma-minenadora-determina-que-sua-funuxe7uxe3o-de-custo-total-para-a-extrauxe7uxe3o-de-certo-tipo-de-ferro-uxe9-dada-por-cx25x2432x-1200-em-r-em-que-x-uxe9-dada-em-toneladas-de-ferro.-determine-o-custo-marginal-da-mineradora.}{%
\section{\texorpdfstring{9. Uma minenadora determina que sua função de
custo total para a extração de certo tipo de ferro é dada por
\(C(x)=2,5x^2+4,32x +1200\) em R\$, em que \(x\) é dada em toneladas de
ferro. Determine o custo marginal da
mineradora.}{9. Uma minenadora determina que sua função de custo total para a extração de certo tipo de ferro é dada por C(x)=2,5x\^{}2+4,32x +1200 em R\$, em que x é dada em toneladas de ferro. Determine o custo marginal da mineradora.}}\label{uma-minenadora-determina-que-sua-funuxe7uxe3o-de-custo-total-para-a-extrauxe7uxe3o-de-certo-tipo-de-ferro-uxe9-dada-por-cx25x2432x-1200-em-r-em-que-x-uxe9-dada-em-toneladas-de-ferro.-determine-o-custo-marginal-da-mineradora.}}

\(C(x)=2,5x^2+4,32x +1200\)

\(C'(x)=5x+4,32\) (\(Cmg\))

\hypertarget{determine-as-derivadas-parciais-1uxaa-e-2uxaa-ordens-de-fxyx3x2y3-2y2.-obtenha-f_x21-e-f_y21}{%
\section{\texorpdfstring{11. Determine as derivadas parciais (1ª e 2ª
ordens) de \(F(x,y)=x^3+x^2y^3-2y^2\). Obtenha \(F_x(2,1)\) e
\(F_y(2,1)\)}{11. Determine as derivadas parciais (1ª e 2ª ordens) de F(x,y)=x\^{}3+x\^{}2y\^{}3-2y\^{}2. Obtenha F\_x(2,1) e F\_y(2,1)}}\label{determine-as-derivadas-parciais-1uxaa-e-2uxaa-ordens-de-fxyx3x2y3-2y2.-obtenha-f_x21-e-f_y21}}

\(F(x,y)=x^3+x^2y^3-2y^2\)

\begin{quote}
\(F_x(x,y)=3x^2+2xy^3\)
\end{quote}

\begin{quote}
\begin{quote}
\(F_{xx}(x,y)=6x+2y^3\)
\end{quote}
\end{quote}

\begin{quote}
\begin{quote}
\(F_{xy}(x,y)=6xy^2\)
\end{quote}
\end{quote}

\begin{quote}
\(F_y(x,y)=3x^2y^2-4y\)
\end{quote}

\begin{quote}
\begin{quote}
\(F_{yx}(x,y)=6xy^2\)
\end{quote}
\end{quote}

\begin{quote}
\begin{quote}
\(F_{yy}(x,y)=6x^2y-4\)
\end{quote}
\end{quote}

\(F_x(2,1)=3.2^2+2.2.1^3\)

\(F_x(2,1)=16\)

\(F_y(2,1)=3.2^2.1^2-4.1\)

\(F_y(2,1)=8\)

\hypertarget{seja-u4x3y2-a-funuxe7uxe3o-que-da-a-utilidade-de-um-consumidor-de-dois-produtos-de-quantidades-x-e-y.-calcule-as-derivadas-parciais-de-primeira-ordem.}{%
\section{\texorpdfstring{13. Seja \(U=4x^3y^2\) a função que da a
utilidade de um consumidor de dois produtos de quantidades \(x\) e
\(y\). Calcule as derivadas parciais de primeira
ordem.}{13. Seja U=4x\^{}3y\^{}2 a função que da a utilidade de um consumidor de dois produtos de quantidades x e y. Calcule as derivadas parciais de primeira ordem.}}\label{seja-u4x3y2-a-funuxe7uxe3o-que-da-a-utilidade-de-um-consumidor-de-dois-produtos-de-quantidades-x-e-y.-calcule-as-derivadas-parciais-de-primeira-ordem.}}

\(U=4x^3y^2\)

\(U_x=12x^2y^2\)

\(U_y=8x^3y\)
